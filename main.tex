% Welcome, Math for Math's sake folks!
%
% My idea is to use this file way to share solutions
% questions, and comments.
%
% Feel free to edit any part of it, and add
% ``Your name:your comment, question, or solution'' so we know
% who is asking what.
%
% You don't have to worry too much about making mistakes in LaTeX
% format: ShareLatex keeps track of the file's history, so it's pretty
% easy to fix mistakes.  There's help here if you want to learn LaTeX
% and ShareLatex: https://www.sharelatex.com/learn
%
% If you're a techy person, this is also linked to a github repo
% here: https://github.com/mkoconnor/Math-for-Math-s-Sake-Logic-chapters-5-6-7-and-maybe-8
% Feel free to submit a pull request.  If you've never heard of github,
% just ignore this paragraph
%
% This is a total experiment, and I'm not sure if will people will
% like or use it, but let's find out!
\documentclass{article}
\usepackage[utf8]{inputenc}

\title{Logic, chapters 5, 6, 7, and (maybe!) 8}
\author{Math for Math's Sake}
\newcommand\s{\section*}
\renewcommand\ss{\subsection*}
\newcommand\sss{\subsubsection*}

\begin{document}
\maketitle
\s{5 Abacus Machines}
\ss{1}
\ss{2}
\ss{3}
\ss{4}
\ss{5}
\ss{6}
\ss{7}
\ss{8}
\ss{9}
\ss{10}
\ss{11}

\s{6 Recursive Functions}
% Some helper commands for the functionals defined in the text
\newcommand\id{\mathrm{id}} % identity
\newcommand\Cn{\mathrm{Cn}} % composition
\newcommand\Pr{\mathrm{Pr}} % primitive recursion
\newcommand\Mn{\mathrm{Mn}} % minimization

\ss{1}
Suppose $f(x,y)$ is recursive.
\sss{a}
Then $g(x,y) = f(y,x)$ is recursive by $g(x,y) = 
\ss{2}
\ss{3}
\ss{4}
\ss{5}
\ss{6}
\ss{7}
\ss{8}
\ss{9}

\s{7 Recursive Sets and Relations}
\ss{1}
\ss{2}
\ss{3}
\ss{5}
\ss{12}
\ss{13}
\ss{14}
\ss{15}
\ss{16}
\ss{17}

\s{8 Equivalent Definitions of Computability}
\ss{1}
\ss{2}
\end{document}
